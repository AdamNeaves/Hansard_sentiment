\thispagestyle{empty}

\begin{center}
    {\LARGE\bf Abstract}
\end{center}

Politics affects all aspects of a persons life. The results of debates in Parliament may have a profound influence on an individuals life, and knowing how your local MP speaks and the opinions they express during these debates could prove useful. Websites, such as \href{https://www.theyworkforyou.com}{They Work For You}, show a user how their local MP votes, and how often they attend debates, ask questions, and other information that may allow the user to make informed decisions when voting during elections.

The aim of the Hansard Sentiment Analysis Tool is to provide a tool to automatically detect the sentiment of statements made during political debates, and to compare it with sentiment expressed about the topic previously, or compare it with sentiment expressed by the same MP. The sentiment analysis will use a machine learning approach, where a model will be trained on labelled datasets generated from the source data.

Being able to view how sentiment changed over time, expecially about certain topics, could prove useful for a user who wants to not only know how an MP votes, but also how they represent themselves and their constituency in parliament. If an MP is seen to change opinion on a topic as time passes, this information could be used to keep voters informed.

This report will document the process of designing and developing the Hansard Sentiment Analysis Tool, highlighting any challenges encountered during development, and also the results of the technical work.
