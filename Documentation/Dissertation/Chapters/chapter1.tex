\chapter{Background \& Objectives}

\section{Background}
In order to prepare for this project, my first task was to download the actual Hansard Dataset to see what I was going to be working with, and to research the purpose and contents of the Hansard archive. Additionally, I began researching Natural Language Processing, seeing what programming languages were commonly used for this purpose and what the standard techniques usually are. In my research, I saw that the most commonly used languages were either Python, or Java. Considering the size of the dataset used and considering speed, I chose to develop this system in Python.

Once the language was chosen, I could look into the sort of packages commonly used in Natural Language Processing, and research sentiment analysis more thoroughly. 

\section{Analysis}
Taking into account the problem and what you learned from the background work, what was your analysis of the problem? How did your analysis help to decompose the problem into the main tasks that you would undertake? Were there alternative approaches? Why did you choose one approach compared to the alternatives? 

There should be a clear statement of the objectives of the work, which you will evaluate at the end of the work. 

In most cases, the agreed objectives or requirements will be the result of a compromise between what would ideally have been produced and what was determined to be possible in the time available. A discussion of the process of arriving at the final list is usually appropriate.

As mentioned in the lectures, think about possible security issues for the project topic. Whilst these might not be relevant for all projects, do consider if there are relevant for your project. Where there are relevant security issues, discuss how they will this affect the work that you are doing. Carry forward this discussion into relevant areas for design, implementation and testing.

\section{Process}
You need to describe briefly the life cycle model or research method that you used. You do not need to write about all of the different process models that you are aware of. Focus on the process model that you have used. It is possible that you needed to adapt an existing process model to suit your project; clearly identify what you used and how you adapted it for your needs.
