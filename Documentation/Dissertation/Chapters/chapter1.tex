\chapter{Background \& Objectives}

\section{Background}

\subsection{Hansard Dataset}

The Hansard Dataset is a set of documents produced by the British Parliament, which began in the 18th and 19th century. These documents contain reports and details of debates in the House of Commons, going back to the year 1803. Eventually, in 1907, these reports were made official and started being produced by Parliament itself, becoming “The Official Report”, though still unofficially known as Hansard. Along with becoming official, a report was officially defined as being one:
\begin{quote} “which, though not strictly verbatim, is substantially the verbatim report, with repetitions and redundancies omitted and with obvious mistakes corrected, but which on the other hand leaves out nothing that adds to the meaning of the speech or illustrates the argument" (PUT REF HERE)
\end{quote}
Hansard is available in a variety of versions. The most commonly used and best known version is the Daily Hansard, which appears each morning and reports of the previous day’s proceedings. However, access to this is via an API that only provides the most recent 7 days. For this project, most training and processing will be done on the Historical Hansard dataset, which is a dataset containing all 6 series of Hansard, between 1803 to 2004, though it is expected that some of the older documents will be less useful for this project due to the likelihood of them using outdated speech that would no longer be relevant.

The Historical Hansard Dataset is available online in an XML format. Multiple documents per series are available, each document covering a few days debates at most. It is a large dataset, reaching around 10Gb in size in total. Most of the documents available are scanned from hard copies, rather than typed up directly, meaning there is a possibility of small errors from the scanning process that may have to be dealt with. Additionally, from preliminary looks, the data itself appears to be only loosely formatted, and each of the six series appear to be formatted in a slightly different way, so any system designed to read this data will have to be capable of dealing with any changes to formatting.

\subsection{Natural language Processing}
To prepare for the project, there were two major areas needing research; Natural Language Processing, and the Hansard Dataset.

Research began on the Hansard Dataset, where the first task was to download the actual dataset, a 10Gb verbatum record of everything said in parliament between 1803 and 2004, but “...with repetitions and redundancies omitted and with obvious mistakes corrected…" (https://www.commonwealth-hansard.org/about-hansard.html)
This is sourced from hard copies, and as such there are some minor errors from when it was scanned, assumably by some form of OCR. However, these errors are minor, and uncommon enough that they prove no issue for the project as a whole.

One aspect of the Hansard dataset that needed to be planned for when designing the project was it’s lack of proper XML formatting. Though the data is presented in an Xml format, there were also HTML style tags within the bodies of text. In addition, each of the six series of data appeared to have slightly different formatting, and so anything used to read the data would have to be able to handle these differences.

In researching Natural language Processing, the Natural language Toolkit was discovered. This python module is a commonly used for Natural langage Processing, and provides methods and classes for many things, such as Named Entity Regongition, sentence splitting, and even some basic machine learning tactics for things like text classification, which could be used for the project. 

\section{Analysis}
Taking into account the problem and what you learned from the background work, what was your analysis of the problem? How did your analysis help to decompose the problem into the main tasks that you would undertake? Were there alternative approaches? Why did you choose one approach compared to the alternatives? 

There should be a clear statement of the objectives of the work, which you will evaluate at the end of the work. 

In most cases, the agreed objectives or requirements will be the result of a compromise between what would ideally have been produced and what was determined to be possible in the time available. A discussion of the process of arriving at the final list is usually appropriate.

As mentioned in the lectures, think about possible security issues for the project topic. Whilst these might not be relevant for all projects, do consider if there are relevant for your project. Where there are relevant security issues, discuss how they will this affect the work that you are doing. Carry forward this discussion into relevant areas for design, implementation and testing.

\section{Process}
You need to describe briefly the life cycle model or research method that you used. You do not need to write about all of the different process models that you are aware of. Focus on the process model that you have used. It is possible that you needed to adapt an existing process model to suit your project; clearly identify what you used and how you adapted it for your needs.
