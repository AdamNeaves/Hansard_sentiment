\documentclass[draft]{article}
\usepackage[utf8]{inputenc}

\author{Adam Neaves (adn2@aber.ac.uk)}
\title{Hansard Historical Sentiment Analysis and Comparison}
\date{\today}
\begin{document}
\maketitle
Degree Scheme: GH7JP, Artificial Intelligence and Robotics\\
Module Code: CS39440\\
Supervisor: Amanda Clare (afc@aber.ac.uk)\\
Version: 1.0 (First Draft)\\

\clearpage
\section{Project Description}
The Hansard  Sentiment Analysis will develop a Python based Sentiment Analysis tool trained on Parliamentary debates, and aims to provide an output comparing the sentiment of modern Parliament with sentiment found in the historical Hansard Dataset.
\\ \\
The Hansard dataset provides a written record of all written and oral debates held in the House of Commons and the House of Lords from between 1803 and 2004, in XML. Using this archive of data, I hope to be able to extract the debates themselves
from the dataset and produce a record of sentiment on a number of topics, which can then be compared to the modern debates, as they are released. In doing this, I should be able to detect any patterns in the change of opinion, hopefully showing trends
in how political opinion has changed over the past 200 years. 
\\ \\
Python 3 has been selected as the language of choice, due to it's accessibility, and the availability of multiple packages that may prove useful for this project, such as BeautifulSoup for XML parsing, an spaCy or NLTK for the Natural Language Processing.

\section{Proposed Tasks}
\subsection{Investigate Hansard Data}
The Hansard Dataset provides full records of debates in an XML format. However, because these are historical documents going back over 200 years, the formatting of the data may be a little messy at times. I will likely have to manually look at examples of the
data to work out how I can confidently extract the relevant data from these files, and which parts are going to be useful to the project. For instance, each file appears to start with a list of the present members of parliament, which may not be useful for my aims.
\subsection{Investigation of XML Parsers for Python}
The provided Hansard Dataset is a series of XML files. These files need to be parsed and tidied by the system before they can be used for Sentiment Analysis. I need to investigate the XML parsers available for Python, and work out which will work best for me.
\subsection{Investigate Sentiment Analysis Tools}
There are a number of Natural Language Processing packages available for Python. I need to examine the options available and choose a package based on factors, such as ease of use, speed, and licensing. I also need to decide on the factors that I will be basing my decision on.
\subsection{Create Training Dataset}
The sentiment analysis tool will be a form of machine learning. In order to efficiently train the system, I will have to provide it with a set of labelled training data, which will have to be created by hand. This may involve the creation of a tool to aid me in labelling the data,
the creation of which should save time in the long run when labelling the data.
\subsection{Train Sentiment Analysis Model}
Once the training and testing datasets have been created, I can train the Sentiment Analysis model on the data. This should produce the model 
\section{Project Deliverables}
\subsection{Sentiment Analysis Model}
A model trained and tested on the Hansard Dataset, capable of detecting the sentiment from the questions and answers provided in the dataset.
\subsection{Training and Testing Datasets}
The datasets used to train and test the Model, which will be provided as an appendix to the report written to show what the model was trained on.
\subsection{Data Sentiment labelling Tool}
if it's deemed appropriate, a tool designed to help manually label sentiment in order to produce the training dataset will be created, and delivered alongside the core software.
\clearpage
\section{Annotated Bibliography}
\bibliography{annotatedBib}
\end{document}
